%% 根据自己实际情况增加或删减
\begin{publications}
\noindent
\textbf{发表与接收论文}
\renewcommand{\labelenumi}{[\arabic{enumi}]}
\begin{enumerate}
    \item 参照参考文献列出学术论文相关信息(含期刊、会议、或参编书稿),但无论有多少个作者,都必须列出全部作者名;若为英文论文,则名在前姓在后,姓名均为全称;在本人的名字加粗,以示区别(若为第一作者,则需在最后特别注明署名华中科技大学是否为第一单位)
    \item 若已发表,按参考文献给出页码;若只是online,给出链接;若接受或修改或投稿或拟投,也必须分别注明
    \item 一般情况,一作或重要的论文放在前面
    \item Hajiali, Faezeh, \textbf{Saeid Tajbakhsh}, and Akbar Shojaei. Fabrication and properties of polycaprolactone composites containing calcium phosphate-based ceramics and bioactive glasses in bone tissue engineering: a review[J]. Polymer Reviews, vol. 58, no. 1, pp. 164-207, 2018.(SCI源刊;IF:5.375;署名单位:华中科技大学)
\end{enumerate}
\vspace{2em}
\textbf{专\hspace{2em}利}
\renewcommand{\labelenumi}{[\arabic{enumi}]}
\begin{enumerate}
    \item 全部作者的姓名全称,本人的名字加粗. 专利题名. 专利国别,专利文献种类,专利号或申请号
\end{enumerate}
\vspace{2em}
\textbf{标\hspace{2em}准}
\renewcommand{\labelenumi}{[\arabic{enumi}]}
\begin{enumerate}
    \item 全部作者的姓名全称,本人的名字加粗. 标准题名. 哪种层次的标准,发表年
\end{enumerate}
\vspace{2em}
\textbf{科技奖励}
\renewcommand{\labelenumi}{[\arabic{enumi}]}
\begin{enumerate}
    \item 全部作者的姓名全称,本人的名字加粗. 题目. 国家级/省部级科技类奖,获奖年
    \item 全部作者的姓名全称,本人的名字加粗. 题目. 国际/国内竞赛类奖,获奖年
\end{enumerate}
\end{publications}
