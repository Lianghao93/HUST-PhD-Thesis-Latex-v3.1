%%% mode: latex
%%% TeX-master: t
%%% End:

\chapter{模板简介}
\label{cha:intro}

\section{概述}
\label{sec:general intro}
论文页边距、行距全文统一。正文小四,中文宋体、英文Times New Roman, 1.5倍行距,图表五号。
非章节结束,正文页不留白,确有需要留白不超过3行。

本模板为华工博士论文~\LaTeX~模板~3.2
版本,新版本基于清华大学学位论文~\LaTeX+CJK
模板(薛瑞尼版本)和华中科技大学博士学位论文~\LaTeX+CJK 模板1.0
版本。模板作者试图尽量使此模板满足华工研院提出的格式要求~\cite{guide,standard},
但不承诺~100\%
满足,即:存在模板使用者需要在此模板基础上再作修正的可能。
本模板亦可充当一份速查手册,文中包含了尽可能多的各种论文常见元素。
但值得注意的是,在正式的论文写作中,应保持格式上的简洁和风格上的
一致,避免出现种类繁多的元素。

\section{参考资料}
\label{sec:requirement}
本文主要为关于~\LaTeX~模板本身使用方法的介绍,关于毕业论文写作内容上的要求,使用者请参阅《华中科技大学研究生学位论文写作指南》~\cite{guide}、《华中科技大学博士、硕士学位论文撰写规定》~\cite{standard}、《理工科-博士-华中科技大学学位论文参考模板》~\cite{modal}等文档。

模板需要使用者具备基本的~\LaTeX~知识,并有一定的使用能力。
对于一些常见构成元素的使用问题,如表格、图形等,
使用者可查阅~\inlinecite{TEXGURU99, OETIKER02} 等文档。
\section{模板编译简介}
\label{sec:compile}

\subsection{XeLatex}
\label{sec:xelatex}

这种编译模式下执行的命令依次为:
\begin{verbatim}
xelatex main 
bibtex main % 编译参考文献文件 *.bib
xelatex main
\end{verbatim}
注意当文档中的引用信息(ref 和 cite)发生变化后,就至少需要运行~
3 次~latex 命令,从而正确显示交叉引用信息。

\section{模板包含文件简介}
\label{sec:checklist}

\subsection{文件清单}

表~\ref{tab:template-files} 列举了本模板主要文件及其功
能。
\begin{table}[htb]
  \centering
  \caption{模板文件清单}
  \label{tab:template-files}
  \begin{minipage}[t]{0.8\linewidth} % 如果想在表格中使用脚注,minipage是个不错的办法
    \begin{tabular*}{\linewidth}{m{3cm}m{10cm}}
      \toprule[1.5pt]
      {\hei 文件名}  & {\hei 描述} \\\midrule[1pt]
      HUSTthesis.cls & 模板类文件\\
      HUSTbib.bst    & 参考文献~Bib\TeX{} 样式文件\\
      HUSTtils.sty   & 常用的包和命令写在这里,减轻主文件的负担\\
      \bottomrule[1.5pt]
    \end{tabular*}
  \end{minipage}
\end{table}

\subsection{目录结构}

\begin{tabular}{l l}
\texttt{figures} & 图像文件\\
\texttt{ref} & 参考文献\\
\texttt{fonts} & 字体文件\\
\texttt{body} & 剩下的文件,包括:封面、各章节、结论、致谢、附录、发表论文列表\\
\end{tabular}

\subsection{宏包清单}

本模板可能用到的所有宏包列举如下:
\begin{enumerate}
\item 字体宏包: arial, helvet, mathptmx, courier, bm
\item 数学环境宏包:amsmath, amssymb
\item 图形宏包:graphicx, subfig
\item 表格宏包:array, booktabs
\item 文本排版宏包:indentfirst, ntheorem, titletoc
\item 引用及辅助宏包:hypernat, natbib, hyperref
\item 其它辅助宏包:ifthen, calc, ifpdf
\end{enumerate}
上面各宏包都包含在最新的TeX Live套装中,如果缺其中的某些宏包,
请下载最新的TeX Live套装。

\section{类文件选项}
\label{sec:options}

本模板相对与原先的~1.0 版本最大的区别是提供了一个类文件~
HUSTthesis.cls,所以使用的时候需要加载相应的类选项。典型的加载方式
如下:

\verb|\documentclass[draftformat,mathCMR]{HUSTthesis}|

\noindent 下面逐一介绍所有选项以及相应的可选值。

\subsection{mathtimes, mathCMR}

公式字体选项,mathtimes 选项让公式启用~Times Roman 字体,mathCMR
选项让公式启用~CM Roman
字体。目前学校尚未规定公式选用什么字体,推荐使用~CM Roman 字体,
因为~Times Roman 数学字体不支持黑体。 如果使用~Times Roman
字体,需加载~bm 宏包用于支持黑体(不推荐)。

\subsection{draftformat, finalformat}

提交草稿打开~draftformat 选项,提交最终版打开~finalformat 选项。
草稿正文页包括页眉(“华中科技大学硕士学位论文”),页眉修饰线(双线),
页脚(页码),页脚修饰线(单线)。
最终版正文页不包括页眉、页眉修饰线和页脚修饰线,仅包含页脚。

\section{更新记录}

\begin{center}
\begin{longtable}{cccp{9cm}}
\hlinewd{1.5pt}
日期 & 版本 & 更新人 & 说明\\
\midrule[0.5pt]
2005/06/22 & 1.00 & Feng Jiang & 首次发布于白云黄鹤~BBS \TeX~版。\\
2006/06/14 & 2.00 & 刘慧侃 & 发布于白云黄鹤~BBS Paper 版。\\
2020/11/16 & 3.00 & Xinze Zhang & 发布于GitHub。\\
2022/03/26 & 3.2  & Lianghao Li \& Jianqing Lin &发布于GitHub。\\
... & ... & ... & ...\\
\hlinewd{1.5pt}
\end{longtable}
\end{center}
